\documentclass[xcolor=dvipsnames, 11pt]{beamer}

\usetheme{Warsaw}

\usepackage{ucs}
\usepackage[utf8x]{inputenc}
\usepackage[greek,english]{babel}
\usepackage{hyperref}
\usepackage{tcolorbox}
\usepackage{alphabeta}
\usepackage{amsmath}
\usepackage{amsthm}

\tcbuselibrary{theorems}

\newtcbtheorem{mytheorem}{\el Ορισμός}%
{colback=blue!5,colframe=black!15!black,fonttitle=\bfseries}{th}

\newtcbtheorem{mytheor}{\el Θεώρημα}%
{colback=blue!5,colframe=black!15!black,fonttitle=\bfseries}{th}

\newcommand{\en}{\selectlanguage{english}}
\newcommand{\el}{\selectlanguage{greek}}

\setbeamertemplate{itemize items}[ball]
\setbeamertemplate{itemize subitem}[ball]

\begin{document}

\title{Travelling Salesman Problem}
\subtitle{TSP}

\author[\el Σιώρος Βασίλειος, Ανδρινοπούλου Χριστίνα]{\el Σιώρος Βασίλειος \and \el Ανδρινοπούλου Χριστίνα}
\date{\el Απρίλιος, 2020}

\frame{\titlepage}

\begin{frame}
	\frametitle{\el Ορισμός}
	\begin{itemize}
		\item \el ..
	\end{itemize}
\end{frame}

\begin{frame}
	\frametitle{\el Είδη \en TSP}
	\begin{itemize}
		\item \el ..
	\end{itemize}
\end{frame}

\begin{frame}
	\frametitle{\el Εφαρμογές}
	\begin{itemize}
		\item \el αρα και γιατι ειναι σημαντικο ως προβλημα
		\item \el αρα και γιατι το επιλεξαμε
	\end{itemize}
\end{frame}

\begin{frame}
	\frametitle{\el Προσεγγίσεις}
	\begin{itemize}
		\item \el εδώ μπορουμε να πουμε και για την υλοποιηση
	\end{itemize}
\end{frame}


\begin{frame}
	\frametitle{References}
	\begin{itemize}
		\item \textit{\en Exact and Approximation Algorithms for Time-Window TSP, 
			Jie Gao, Su Jia, Joseph S. B. Mitchell,
			CG:YRF, Boston, MA, USA, June 14-18, 2016}
		\item \textit{\en An Optimal Lower Bound for the Hilbert-type, Planar Universal Traveling Salesman Problem, Patrick Eades, Julián Mestre,	CG:YRF, Brisbane, Australia, July 4-7, 2017}
		\item \textit{\en The Geometric Maximum Traveling Salesman Problem,	David S. Johnson, Arie Tamir,
			Article in Journal of the ACM · May 2002}	
\end{itemize}
\end{frame}

\begin{frame}
	\frametitle{References}
	\begin{itemize}
		\item \textit{\el Εισαγωγή στους αλγορίθμους, Δεύτερη έκδοση, \en Thomas H. Cormen, Charles E. Leiserson, Ronald L. Rivest, Clifford Stein, \el Πανεπιστημιακές εκδόσεις Κρήτης, 2011, \en ISBN: 978-960-524-473-6}
		\item \textit{\el Τεχνητή Νοημοσύνη, Μία σύγχρονη προσέγγιση, Δεύτερη Αμερικανική έκδοση, \en Stuart Russel, Peter Norvig, \el σελ.: 101,	Κλειδάριθμος 2005, \en ISBN: 960-209-873-2}
		\item \textit{\el Στοιχεία διακριτών μαθηματικών, \en C. L. Liu, \el σελ.: 171-172, 178-179, 190-201, Πανεπιστημιακές εκδόσεις Κρήτης 2014, \en ISBN: 978-960-524-072-1}	
	\end{itemize}
\end{frame}

\begin{frame}
	\frametitle{References}
	\begin{itemize}
		\item \textit{\el Discrete and Computational Geometry, Satyan L. Devadoss, Joseph O'Rourke,
			σελ.: 81-86, Princeton University Press, 2011, ISBN: 978-0-691-14553-2}
		\item \textit{\el Computational Geometry,	Algorithms and Applications, Third Edition, 
			Mark de Berg, Otfried Cheong, Marc van Kreveld, Mark Overmars, σελ.: 193-204,
			Springer, 2008, ISBN: 978-3-540-77973-5}
		\item \textit{\el Υπολογιστική Γεωμετρία: Μια σύγχρονη αλγοριθμική προσέγγιση, Γιάννης Ζ. Εμίρης,
			σελ.: 199-208, Κλειδάριθμος, 2008, ISBN: 978-960-461-141-6 }	
	\end{itemize}
\end{frame}

\end{document}