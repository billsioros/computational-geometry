\documentclass[xcolor=dvipsnames, 11pt]{beamer}

\usetheme{Warsaw}

\usepackage{ucs}
\usepackage[utf8x]{inputenc}
\usepackage[greek,english]{babel}
\usepackage{hyperref}
\usepackage{tcolorbox}
\usepackage{alphabeta}
\usepackage{amsmath}
\usepackage{amsthm}

\tcbuselibrary{theorems}

\newtcbtheorem{mytheorem}{\el Ορισμός}%
{colback=blue!5,colframe=black!15!black,fonttitle=\bfseries}{th}

\newtcbtheorem{mytheor}{\el Θεώρημα}%
{colback=blue!5,colframe=black!15!black,fonttitle=\bfseries}{th}

\newcommand{\en}{\selectlanguage{english}}
\newcommand{\el}{\selectlanguage{greek}}

\setbeamertemplate{itemize items}[ball]
\setbeamertemplate{itemize subitem}[ball]

\begin{document}

\title{Travelling Salesman Problem}
\subtitle{TSP}

\author[\el Σιώρος Βασίλειος, Ανδρινοπούλου Χριστίνα]{\el Σιώρος Βασίλειος \and \el Ανδρινοπούλου Χριστίνα}
\date{\el Απρίλιος, 2020}

\frame{\titlepage}

\begin{frame}
	\frametitle{\el Ορισμός}
	\begin{itemize}
		\item \el Κλασσικό πρόβλημα της θεωρητικής επιστήμης των  υπολογιστών. 
		\item \el Ο πωλητής οφείλει να επισκεφτεί \en n \el το πλήθος πόλεις για να πουλήσει το εμπόρευμά του.  
		\item \el Ο πωλητής πρέπει να επισκεφτεί την κάθε πόλη ακριβώς μία φορά ακολουθώντας το συντομότερο δρομολόγιο και να επιστρέψει στην πόλη εκκίνισης.
	\end{itemize}
	\begin{figure}
		\includegraphics[scale=0.3]{images/tsp.png}
		\caption{\en Travelling Salesman Problem - TSP}
	\end{figure}
\end{frame}

\begin{frame}
	\frametitle{\el Εφαρμογές}
	\begin{itemize}
		\item \el Μεταφορές και \en Logistics:
		\begin{itemize}
			\item \el Καθορισμός δρομολογίων σχολικών λεωφορείων
			\item \el Παράδοση τροφίμων σε άτομα που δεν μπορούν να μετακινηθούν από το σπίτι
			\item \el Δρομολόγηση φορτηγών για παραλαβή δεμάτων
			\item \el Εφοδιασμός σε ορόφους καταστημάτων ή σε αποθήκες
		\end{itemize}
		\item \el Βιολογία:
		\begin{itemize}
			\item \en DNA sequencing: οι "πόλεις" είναι DNA strings και η απόσταση είναι τα αποτελέσματα μέτρων σημασιολογικής ομοιότητας
		\end{itemize}
		\item \el Διάστημα:
		\begin{itemize}
			\item \el Ελαχιστοποίηση των καυσίμων που απαιτούνται για την παρατήρηση ουράνιων αντικειμένων
		\end{itemize}	
	\end{itemize}
\end{frame}

\begin{frame}
	\frametitle{\el Μελέτη}
	\begin{itemize}
		\item \en Time Window TSP
		\begin{itemize}
			\item \el Κάθε σημείο επίσκεψης \(v\) \el έχει ένα χρονικό \en deadline \(D(v)\).
			\item \el Κάθε σημείο επίσκεψης \(v\) χαρακτηρίζεται από έναν χρόνο ελευθέρωσης \en (release time) \(R(v)\).
			\item \el Ο πωλητής μπορεί να επισκεφτεί ένα σήμειο μόνο εντός του χρονικού πλαισίου \([R(v), D(v)]\).
		\end{itemize}
		\item Universal TSP
		\begin{itemize}
			\item \el Κάθε μέρα ο πωλητής πρέπει να επισκέπτεται ένα διαφορετικό υποσύνολο πόλεων.
			\item \el Δημιουργία ενός κατάλληλου \en master tour \el που περιλαμβάνει όλες τις πόλεις.
			\item \el Προσαρμογή του καθημερινού δρομολογίου με βάση το \en master tour \el σύμφωνα με τις εκάστοτε ανάγκες.
			\item \el Στόχος: Δημιουργία κατάλληλου \en master tour \el, ώστε τα καθημερινά δρομολόγια να είναι αποδοτικά. 
		\end{itemize}
	\end{itemize}
\end{frame}


\begin{frame}
	\frametitle{References}
	\begin{itemize}
		\item \textit{\en Exact and Approximation Algorithms for Time-Window TSP, 
			Jie Gao, Su Jia, Joseph S. B. Mitchell,
			CG:YRF, Boston, MA, USA, June 14-18, 2016}
		\item \textit{\en An Optimal Lower Bound for the Hilbert-type, Planar Universal Traveling Salesman Problem, Patrick Eades, Julián Mestre,	CG:YRF, Brisbane, Australia, July 4-7, 2017}
		\item \textit{\en An O(nlogn) Heuristic for the Euclidean Traveling Salesman Problem, 
			Evgeny Yanenko, Eckart
Schuhmacher, Ulrich Spörlein, Kurt Tutschku,
			April 25, 2005}	
		\item \text{\en Applications of the TSP,} \\
			 \text{http://www.math.uwaterloo.ca/tsp/apps/index.html}
\end{itemize}
\end{frame}

\end{document}