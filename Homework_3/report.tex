
\documentclass[12pt]{article}

\usepackage[utf8]{inputenc}
\usepackage[greek, english]{babel}

% Packages
\usepackage{alphabeta}
\usepackage{amsmath}
\usepackage{amsthm}
\usepackage{caption}
\usepackage{color}
\usepackage{float}
\usepackage{fullpage}
\usepackage{graphicx}
\usepackage{hyperref}
\usepackage{latexsym}
\usepackage{listings}
\usepackage{pxfonts}
\usepackage{stackrel}
\usepackage{subfig}
\usepackage{tikz}
\usepackage{titlesec}

% Commands
\newcommand{\N}{\mathbb{N}}
\newcommand{\R}{\mathbb{R}}
\newcommand{\abs}[1]{\left\lvert#1\right\rvert}
\newcommand{\code}[2]{\lstinputlisting[caption={#2}]{#1}}
\newcommand{\margin}{\hspace{4pt}}
\newcommand{\norm}[1]{\left\lVert#1\right\rVert}

% Environments
\newenvironment{matlab}
	{\begin{figure}[H]\centering\captionsetup{justification=centering}}
	{\end{figure}}

\newenvironment{rcases}
	{\left.\begin{aligned}}
	{\end{aligned}\right\rbrace}

% Python Syntax Highlighting
\definecolor{string_color}{RGB}{0, 161, 13}
\definecolor{comment_color}{RGB}{46, 46, 46}
\definecolor{keyword_color}{RGB}{0, 112, 191}
\definecolor{background_color}{RGB}{250, 250, 250}

\lstset{
    framesep=15pt,
    xleftmargin=15pt,
    xrightmargin=15pt,
    language=Python,
    captionpos=b,
    numbers=right,
    numberstyle=\small\ttfamily,
    frame=lines,
    showspaces=false,
    showtabs=false,
    breaklines=true,
    showstringspaces=false,
    breakatwhitespace=true,
    commentstyle=\color{comment_color}\textit,
    keywordstyle=\bfseries\color{keyword_color}\textbf,
    stringstyle=\color{string_color}\textit,
    morekeywords={self, lambda, __init__, __del__, __name__, for, in, not, and, or, :},
    basicstyle=\small\ttfamily,
    tabsize=4,
    keepspaces=true,
    columns=flexible,
    backgroundcolor=\color{background_color}
}

% Links
\hypersetup{
    colorlinks=true,
    linkcolor=blue,
    filecolor=magenta,
    urlcolor=cyan,
}

% Lengths
\setlength{\parindent}{0in}
\setlength{\oddsidemargin}{0in}
\setlength{\textwidth}{6.5in}
\setlength{\textheight}{10in}
\setlength{\topmargin}{-1.0in}
\setlength{\headheight}{18pt}

\titlespacing*{\subsection}
{0pt}{5.5ex plus 1ex minus .2ex}{4.3ex plus .2ex}

\title{\hugeΥπολογιστική Γεωμετρία\\Πρώτη Εργασία}
\author{Σιώρος Βασίλειος - 1115201500144\\Ανδρινοπούλου Χριστίνα - 1115201500006}
\date{Ιούνιος 2020}

\begin{document}

\maketitle

\pagenumbering{gobble}

\pagebreak


\subsection*{1. Compute Voronoi diagrams of different sets of vertices of your choice using
the routine Voronoi (and its companion voronoi plot 2d for visualization) from the module
scipy.spatial. Plot your results.}

\vspace{2in}

\pagebreak

\subsection*{2. Using the routine Delaunay in the module scipy.spatial compute the Delaunay
triangulation of different sets of vertices of your choice and plot your results.}

\vspace{2in}

\pagebreak

\subsection*{3. Compute the shortest path of different set of vertices of your choice in a tri-
angulation. By a path in this setting, we mean a chain of edges of this triangulation. Use the
methods in the package scipy.sparse.csgraph.}

\vspace{2in}

\pagebreak

\subsection*{4. Experiment yourself with the .encloses point and .encloses methods of the
sympy.geometry module usingf polygons or circles to check if they contain certain points of
your choice. Do the same with contains point or contains points from the Path class from the
libraries of matplotlib.path.}

\vspace{2in}

\pagebreak

\subsection*{5. The problem of finding the Voronoi cell that contains a given location is
equivalent to the search for the nearest neighbor. We can always perform this search with a
brute force algorithm, but in general there are more elegant and less complex approaches to
this problem like the kd-trees. In the scipy use the class KDTree to perform some experiments
of your choice.}

\vspace{2in}

\pagebreak

\end{document}
